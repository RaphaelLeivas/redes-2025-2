
\chapter{Ambiente de Desenvolvimento}

O projeto é desenvolvido em um notebook Lenovo com processador Intel-Core i5
de 11º geração, com arquitetura x64. A frequência do processador é 2.4 GHz e possui 4 cores. 
O processador possui 8 GB de memória RAM disponível.
O sistema operacional do computador é Windows 11, mas o ambiente NS-3 é instalado 
através de WSL (Windows Subsystem for Linux) Ubuntu 22.04.1 LTS.

O ambiente NS-3 é instalado na pasta \verb|home| do WSL.
A Figura \ref{fig:ambiente-dev} ilustra o ambiente de desenvolvimento 
utilizado no projeto.

\begin{figure}[htb]
	\centering 
    \caption{Ambiente de desenvolvimento.}
	\includegraphics[width=\textwidth]{images/ambiente-dev.png}
	\label{fig:ambiente-dev}
	\fonte{Elaboração própria.}
\end{figure}

Essencialmente, é adotado um procedimento de duas etapas:

\begin{enumerate}
    \item O comando \verb|./ns3 build|, seguido de \verb|./ns3 run| com o programa 
    e argumentos de linha de comando desejados, é usado para executar o script C++ 
    com as funções do NS-3. A saída dos comandos será um arquivo XML;
    \item O comando \verb|./netanim| é usado para visualizar o arquivo XML exportado na etapa 1, obtendo a animação 
    exibida no lado direito da Figura \ref{fig:ambiente-dev}.
\end{enumerate}

\chapter{Parte 1: Topologia com Gargalo}

O primeiro passo é obter uma topologia em que há um link de gargalo 
entre os nós que estão se comunicando. A Figura \ref{fig:topologia-lab2-p1} (a)
mostra a topologia e os tempos de envio e recepção de um pacote (b) e (c). 

\begin{figure}[htb]
	\centering 
    \caption{Topologia base e resultados iniciais da parte 1.}
	\includegraphics[width=\textwidth]{images/topologia-lab2-p1.png}
	\label{fig:topologia-lab2-p1}
	\fonte{Elaboração própria.}
\end{figure}

Na topologia, os links entre os nós 0 - 1 e 2 - 3 possuem largura de banda de 100 Mbps. O link entre 1 - 2 representa
o gargalo, com apenas 1Mbps de banda. Isso se reflete nas Figuras \ref{fig:topologia-lab2-p1} (b) e (c), nas quais os pacotes 
gastam poucos microsegundos para realizar a transmissão entre os nós 0 - 1 e 2 - 3, e mais de 200 milissegundos 
entre os nós de gargalo da rede. 

\section{Janela de Congestionamento}

A ação da janela de congestionamento, quando aplicamos um Bulksend com 5 fluxos simultaneamente pode ser vista na 
Figura \ref{resultado-lab2-p1}. Vemos que quando o gargalo tem banda de 1 Mbps, o Goodput (taxa de dados 
que efetivamente são transmitidos entre o recipiente e o destinatário na rede) fica na ordem de 0.1 a 0.2 Mbps. 
Quando aumentamos a banda do gargalo para 10 Mbps, eles se estabiliza em 1 a 2 Mbps, cerca de 10\% do total do gargalo.
Essas variações nos valores do Goodput refletem as variações na janela de congestionamento do TCP.

\begin{figure}[htb]
	\centering 
    \caption{Ação da janela de congestionamento na topologia com o Bulksend.}
	\includegraphics[width=\textwidth]{images/resultado-lab2-p1.png}
	\label{fig:resultado-lab2-p1}
	\fonte{Elaboração própria.}
\end{figure}

\section{Parte 1a: Comparação entre o TCP CUBIC e o NewReno}