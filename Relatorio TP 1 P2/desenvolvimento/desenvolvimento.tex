
\chapter{Ambiente de Desenvolvimento}

O projeto é desenvolvido em um notebook Lenovo com processador Intel-Core i5
de 11º geração, com arquitetura x64. A frequência do processador é 2.4 GHz e possui 4 cores. 
O processador possui 8 GB de memória RAM disponível.
O sistema operacional do computador é Windows 11, mas o ambiente NS-3 é instalado 
através de WSL (Windows Subsystem for Linux) Ubuntu 22.04.1 LTS.

O ambiente NS-3 é instalado na pasta \verb|home| do WSL.
A Figura \ref{fig:ambiente-dev} ilustra o ambiente de desenvolvimento 
utilizado no projeto.

\begin{figure}[htb]
	\centering 
    \caption{Ambiente de desenvolvimento.}
	\includegraphics[width=\textwidth]{images/ambiente-dev.png}
	\label{fig:ambiente-dev}
	\fonte{Elaboração própria.}
\end{figure}

Essencialmente, é adotado um procedimento de duas etapas:

\begin{enumerate}
    \item O comando \verb|./ns3 build|, seguido de \verb|./ns3 run| com o programa 
    e argumentos de linha de comando desejados, é usado para executar o script C++ 
    com as funções do NS-3. A saída dos comandos será um arquivo XML;
    \item O comando \verb|./netanim| é usado para visualizar o arquivo XML exportado na etapa 1, obtendo a animação 
    exibida no lado direito da Figura \ref{fig:ambiente-dev}.
\end{enumerate}

\chapter{Parte 1: Topologia com Gargalo}

O primeiro passo é obter uma topologia em que há um link de gargalo 
entre os nós que estão se comunicando. A Figura \ref{fig:topologia-lab2-p1} (a)
mostra a topologia e os tempos de envio e recepção dos pacotes (b). 

\begin{figure}[htb]
	\centering 
    \caption{(a) Topologia base da parte 1 e (b) envio de pacotes.}
	\includegraphics[width=\textwidth]{images/topologia-lab2-p1.png}
	\label{fig:topologia-lab2-p1}
	\fonte{Elaboração própria.}
\end{figure}

Na topologia, os links entre os nós 0 - 1 e 2 - 3 possuem largura de banda de 100 Mbps. O link entre 1 - 2 representa
o gargalo, com apenas 1Mbps de banda. Isso se reflete na Figura \ref{fig:topologia-lab2-p1} (a), com um grande número 
de pacotes acumulado nesse ramo. Além disso, na Figura \ref{fig:topologia-lab2-p1} (b) vemos que
os pacotes rapidamente trafegam entre os nós 0 - 1 e 2 - 3, mas gastam muito mais 
tempo no ramo 1 -2, que representa o gargalo.

A Figura \ref{fig:topologia-lab2-p1} (b) mostra também o primeiro comando de ACK do protocolo TCP 
enviado, seguido de rajada de pacotes enviada pelo \verb|BulkSend| configurado no NS-3.

\section{Janela de Congestionamento}

Para avançar na parte 1, é necessário extrair os valores das janelas de congestionamento para realizar as análises na simulação. 
Isso pode ser feito de duas formas:

\begin{itemize}
	\item Usar o procedimento de \verb|fifth.cc|, com o método \verb|TraceConnectWithoutContext| que recebe um callback 
	e nesse callback podemos chamar uma função de log; 
	\item Usar o procedimento de \verb|tcp-variants-comparison.cc|, que conecta ao socket do nó a partir do método 
	\verb|Config::Connect|.
\end{itemize}

Executando ambos arquivos acima sem modificações, eles conseguem extrair os valores da janela de congestionamento. No entanto,
ao adaptar o código desses arquivos para a nova topologia, não foi possível extrair esses valores. Ou nada era printado, ou 
erros de referenciamento de ponteiro como \verb| <Signals.SIGABRT: 6>| aconteciam. Tentei solucionar, sem sucesso. 

Como a parte dois não precisa da janela de congestionamento, vamos avançar para ela nesse momento.

\chapter{Parte 2: Topologia com Gargalo + Fluxos Heterogêneos}

A topologia usada nesse exercício está exibida na Figura \ref{fig:topologia-lab2-p2}. 
O nó 0 (remetente) deseja enviar vários pacotes para os nós 3 e 4 (destinatários). 
Entre os nós 1 e 2, há um gargalo com banda 1 Mbps e 20 ms de delay. A banda 
das demais conexões é 100 ms com delay 0.01 ms, exceto o ramo entre os nós 2 e 3 que possui 
delay consideravelmente maior de 50 ms.

\begin{figure}[htb]
	\centering 
    \caption{Topologia da parte 2.}
	\includegraphics[width=\textwidth]{images/topologia-lab2-p2.png}
	\label{fig:topologia-lab2-p2}
	\fonte{Elaboração própria.}
\end{figure}

Note na Figura \ref{fig:topologia-lab2-p2} que os pacotes vão se acumulando no gargalo.
O objetivo é verificar no experimento como isso afeta a transmissão de dados de entre o remetente 
e os destinatários, inclusive para diferentes algoritmos de controle de congestionamento.
Para isso, será adotado o seguinte procedimento: 

\begin{enumerate}
	\item Calcula o goodput (taxa de dados efetivamente enviada) dos nós 3 e 4; 
	\item Repete 10 vezes e toma a média;
	\item Repete para os algoritmos TCP New Reno e TCP CUBIC;
	\item Faz isso para o número de fluxos de 2, 4, 6 e 8.
	\item Exporta tudo em csv e plota gráficos com os dados coletados em python com a
	ferramenta \verb|matplotlib|.
\end{enumerate}

Calculando o goodput na topologia da Figura \ref{fig:topologia-lab2-p2} com 4 fluxos, obtemos 
os resultados da Figura \ref{fig:result-goodput-lab2p2}. Vemos que o servidor B (nó 4) 
recebe menos dados devido ao maior delay no seu ramo; contudo, devido ao congestionamento 
no ramo central, ele acaba recebendo uma quantidade de dados não muito menor que o nó 3, 
que possui um delay mais de 100 vezes menor. 

Além disso, note que a banda de ambos os ramos é 100 Mbps. No entanto, com o 
gargalo de 1 Mbps entre o remente e o destinatário, a banda efetivamente usada é menos que 1\% disso. 

\begin{figure}[htb]
	\centering 
    \caption{Goodput calculados na topologia da parte 2.}
	\includegraphics[width=\textwidth]{images/result-goodput-lab2p2.png}
	\label{fig:result-goodput-lab2p2}
	\fonte{Elaboração própria.}
\end{figure}

Repetindo esse experimento 10 vezes, obtemos a distribuição de Goodput calculados para cada 
destinatário exibido na Figura \ref{fig:goodput-hist-lab2p2}. Calculando a média de cada um, obtemos 

\[ \mu_A = 0.23 \un{Mbps} \quad , \quad \mu_B = 0.17 \un{Mbps} \]

\begin{figure}[htb]
	\centering 
    \caption{Histograma do goodput obtido por servidor executando 10 vezes.}
	\includegraphics[width=\textwidth]{images/goodput-hist-lab2p2.png}
	\label{fig:goodput-hist-lab2p2}
	\fonte{Elaboração própria.}
\end{figure}

O próximo passo é comparar o goodput médio obtido entre cada servidor para diferentes 
algoritmos de congestionamento, para diferentes números de fluxos.
O resultado obtido pode ser visto na Figura \ref{fig:result-lab2p2}.

\begin{figure}[htb]
	\centering 
    \caption{Diferença de Goodput (Mbps) entre servidores para os diferentes algoritmos.}
	\includegraphics[width=\textwidth]{images/result-lab2p2.png}
	\label{fig:result-lab2p2}
	\fonte{Elaboração própria.}
\end{figure}

Analisando as curvas, podemos chegar às seguintes conclusões:

\begin{itemize}
	\item Ambos algoritmos de congestionamento apresentam Goodput similar entre os mesmos 
	servidores;
	\item O servidor B, em ambos protocolos, apresenta menor Goodput devido ao fato de possuir maior 
	delay que o servidor A; 
	\item Quanto maior o número de fluxos menor o Goodput em todos os casos. Isso ocorre devido ao 
	aumento do congestionamento no ramo de gargalo , reduzindo a quantidade de pacotes que conseguem 
	chegar aos destinatários.
\end{itemize}

Um outra conclusão que podemos tomar a partir do gráfico se refere à como ambos algoritmos tentam ser 
"justos" (fairness) com ambos servidores, tentando fazer com que ambos tenham o mesmo RTT mesmo que o delay 
dos ramos sejam consideravelmente diferente. Para isso, podemos tomar a diferença entre
os Goodputs dos servidores para cada algoritmo e plotar o resultado, exibido na 
Figura \ref{fig:diff-goodputs-lab2p2}.

\begin{figure}[htb]
	\centering 
    \caption{Histograma do goodput obtido por servidor executando 10 vezes.}
	\includegraphics[width=\textwidth]{images/diff-goodputs-lab2p2.png}
	\label{fig:diff-goodputs-lab2p2}
	\fonte{Elaboração própria.}
\end{figure}

Em média, o New Reno apresenta diferença de 0.06 Mbps, enquanto o CUBIC apresenta diferença média 
de 0.1 Mbps. Assim, concluímos que o New Reno é mais justo que o TCP CUBIC, por apresentar menor diferença entre os 
Goodputs de cada servidor.


\chapter{Conclusão}

Tendo em vista os objetivos do exercício, foi possível verificar experimentalmente 
o impacto que gargalos têm na taxa de transmissão global da rede. Além disso, 
algoritmos de controle de congestionamento que havíamos somente visto em sala de aula 
de forma teórica ficaram mais claros, verificando também experimentalmente as diferenças 
entre eles.

Dificuldades encontradas com a extração da janela de congestionamento não 
permitiram aprender muito com o item 1 do exercício. Contudo, dado que na primeira 
parte do TP 1 usamos apenas UDP, foi interessante ver os pacotes TCP na Figura 
\ref{fig:topologia-lab2-p1} (b), em particular o ACK.