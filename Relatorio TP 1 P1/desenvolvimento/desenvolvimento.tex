
\chapter{Ambiente de Desenvolvimento}

O projeto é desenvolvido em um notebook Lenovo com processador Intel-Core i5
de 11º geração, com arquitetura x64. A frequência do processador é 2.4 GHz e possui 4 cores. 
O processador possui 8 GB de memória RAM disponível.
O sistema operacional do computador é Windows 11, mas o ambiente NS-3 é instalado 
através de WSL (Windows Subsystem for Linux) Ubuntu 22.04.1 LTS.

O ambiente NS-3 é instalado na pasta \verb|home| do WSL.
A Figura \ref{fig:ambiente-dev} ilustra o ambiente de desenvolvimento 
utilizado no projeto.

\begin{figure}[htb]
	\centering 
    \caption{Ambiente de desenvolvimento.}
	\includegraphics[width=\textwidth]{images/ambiente-dev.png}
	\label{fig:ambiente-dev}
	\fonte{Elaboração própria.}
\end{figure}

Essencialmente, é adotado um procedimento de duas etapas:

\begin{enumerate}
    \item O comando \verb|./ns3 build|, seguido de \verb|./ns3 run| com o programa 
    e argumentos de linha de comando desejados, é usado para executar o script C++ 
    com as funções do NS-3. A saída dos comandos será um arquivo XML;
    \item O comando \verb|./netanim| é usado para visualizar o arquivo XML exportado na etapa 1, obtendo a animação 
    exibida no lado direito da Figura \ref{fig:ambiente-dev}.
\end{enumerate}

\chapter{Parte 1: Comunicação Ponto-a-Ponto}

Para a parte 1, o objetivo é estabelecer conexões ponto-a-ponto entre clientes 
e um servidor. Usando \verb|nPackets = 2| e \verb|nClients = 3| nos argumentos 
da linha de comando, é obtido a topologia exibida 
na Figura \ref{fig:topologia-p1}.

\begin{figure}[htb]
	\centering 
    \caption{Topologia do experimento da parte 1.}
	\includegraphics[width=0.8\textwidth]{images/topologia-p1.png}
	\label{fig:topologia-p1}
	\fonte{Elaboração própria.}
\end{figure}

A Figura \ref{fig:resultado-p1} mostra os envios dos pacotes dos nós 
clientes para o nó servidor ao longo do tempo, com três clientes e cada um enviando 
dois pacotes.
 
\begin{figure}[htb]
	\centering 
    \caption{Resultado do experimento da parte 1.}
	\includegraphics[width=0.8\textwidth]{images/resultado-p1.png}
	\label{fig:resultado-p1}
	\fonte{Elaboração própria.}
\end{figure}

É possível observar que o tempo de propagação é muito menor do que o tempo de espera 
para envio dos pacotes. O tempo de RTT é de 3.69 ms, enquanto 1 segundo é esperado para 
enviar cada pacote. Em cada envio, o cliente envia para o nó 3 (servidor), e o servidor 
responde de volta para o nó cliente (0 a 2) que o requisitou.

Para tornar o gráfico mais didático, é possível aumentar o tempo de propagação
para 200 ms e reduzir a largura de banda para 5 kbps, 
obtendo o resultado da Figura \ref{fig:resultado-p1-mod}.
Vemos que o pacote gasta mais tempo para sair do cliente e chegar ao servidor, e 
que o servidor demora 1.88 segundos a enviar a resposta ao cliente. 
Esse valor de atraso é dado pela expressão que vimos em sala de aula:

\[ T_{prop} + \frac{tamanho}{banda} =  0.2 + \frac{1024 \cdot 8 \un{bits}}{5 \cdot 10^3 \un{kpbs}} = 1.83  \un{s}\]

\noindent uma vez que o pacote enviado tem 1024 bytes de tamanho. Vemos que os valores 
calculado e teórico são condizentes.

\begin{figure}[htb]
	\centering 
    \caption{Resultado do experimento da parte 1 - tempo de propagação 200 ms e largura de banda 5 kbps.}
	\includegraphics[width=0.8\textwidth]{images/resultado-p1-mod.png}
	\label{fig:resultado-p1-mod}
	\fonte{Elaboração própria.}
\end{figure}

Na pasta compactada no Moodle, está presente um teste com 5 clientes e cada um 
enviando 4 pacotes, bem como o print com as linhas de comando e arquivo txt.

\chapter{Parte 2: Rede Ethernet}

A topologia implementada na parte 2 está exibida na Figura 
\ref{fig:topologia-p2}. Basicamente, o nó 0 quer se comunicar
com o servidor (nó 6). Para isso, ele usa 4 nós CSMA intermediários 
que enviam a requisição ao servidor final, representando o meio Ethernet. Nós CSMA utilizam o protocolo 
Carrier-Sense Multiple-Access (CSMA) que reduz o risco de colisões entre 
os pacotes enviados simultaneamente por vários nós no mesmo meio.

\begin{figure}[htb]
	\centering 
    \caption{Topologia da parte 2.}
	\includegraphics[width=0.8\textwidth]{images/topologia-p2.png}
	\label{fig:topologia-p2}
	\fonte{Elaboração própria.}
\end{figure}

Utilizando 4 nós CSMA intermediários enviando 10 pacotes, a sequência de envio 
dos pacotes em função do tempo está exibida na Figura \ref{fig:pacotes-p2}.

\begin{figure}[htb]
	\centering 
    \caption{Envio dos pacotes na topologia da parte 2.}
	\includegraphics[width=0.8\textwidth]{images/pacotes-p2.png}
	\label{fig:pacotes-p2}
	\fonte{Elaboração própria.}
\end{figure}

O gráfico do atraso 
de fim-a-fim de uma requisição versus o número do pacote está exibido na Figura
\ref{fig:plot-atraso-p2}. A topologia original é a presente no exemplo \verb|second.cc|
da documentação do NS-3, enquanto a topologia modificada é a exibida na Figura 
\ref{fig:topologia-p2} desenvolvida na parte 2. O gráfico foi obtido a partir do  
seguinte procedimento:

\begin{enumerate}
	\item Modifica os códigos para que eles salvem os atrasos em um arquivo txt 
	com as funções prontas de \verb|TxTrace| e \verb|RxTrace|;
	\item Carrega os arquivos txt em um arquivo Python e faz os plots com a biblioteca 
	\verb|matplotlib|.
\end{enumerate}

\begin{figure}[htb]
	\centering 
    \caption{Atraso fim-a-fim em função do número do pacote para cada uma das topologias.}
	\includegraphics[width=0.8\textwidth]{images/plot-atraso-p2.png}
	\label{fig:plot-atraso-p2}
	\fonte{Elaboração própria.}
\end{figure}

É possível observar que o atraso na topologia original 
é menor do que na modificada. Isso é conforme esperado, uma vez que a 
topologia modificada possui um nó a mais, que aumenta o tempo de propagação.
O fato de o aumento no atraso ser constante ao longo dos pacotes reforça essa 
hipótese.

Além disso, o atraso do primeiro pacote é maior do que os demais. Analisando 
os arquivos PCAP gerados para ambas topologias, vemos que isso é causado pelo 
tempo de inicialização necessário para estabelecer a conexão, como a criação do socket UDP 
e conexão com a porta do servidor. A Figura \ref{fig:atraso-pcap} 
ilustra esse efeito nos arquivos PCAP. Temos um atraso de 0.0349 segundos no 
primeiro pacote, e 0.0169 segundos nos demais.

\begin{figure}[htb]
	\centering 
    \caption{Atraso do primeiro pacote nos arquivos PCAP.}
	\includegraphics[width=\textwidth]{images/atraso-pcap.png}
	\label{fig:atraso-pcap}
	\fonte{Elaboração própria.}
\end{figure}


\chapter{Parte 3: Rede WiFi}

A topologia usada na parte 3 está exibida na Figura \ref{fig:topologia-p3}.
Temos dois pontos de acesso (AP) de duas redes Wi-Fi, de modo que os nós 
de cada rede podem comunicar entre si usando esses APs.

\begin{figure}[htb]
	\centering 
    \caption{Topologia da parte 3.}
	\includegraphics[width=\textwidth]{images/topologia-p3.png}
	\label{fig:topologia-p3}
	\fonte{Elaboração própria.}
\end{figure}

Usando os logs de tempo da Figura \ref{fig:logs-p3}, podemos obter os 
atrasos de envio de pacotes e montar o gráfico da Figura \ref{fig:atraso-p3}.

\begin{figure}[htb]
	\centering 
    \caption{Logs com timestamps de envio e recebimento dos pacotes - parte 3.}
	\includegraphics[width=\textwidth]{images/logs-p3.png}
	\label{fig:logs-p3}
	\fonte{Elaboração própria.}
\end{figure}

\begin{figure}[htb]
	\centering 
    \caption{Gráfico com os atrasos de envios de pacotes na topologia com Wi-Fi.}
	\includegraphics[width=\textwidth]{images/atraso-p3.png}
	\label{fig:atraso-p3}
	\fonte{Elaboração própria.}
\end{figure}

Comparado com os atrasos em nós CSMA na rede Ethernet na Figura \ref{fig:plot-atraso-p2},
verificamos um comportamento similar: o envio do primeiro pacote é mais demorado devido à criação 
de sockets UDP e estabelecimento da conexão. Em seguida, os demais pacotes tem 
delay menor com a conexão já estabelecida. 

As diferenças entre os tempos de delay observados são causadas pela diferença entre os meios
físicos de propagação dos pacotes: 
os nós CSMAs da parte 2 usam a rede cabeada através de Ethernet, enquanto a rede Wi-Fi 
usa o meio sem fio com ondas eletromagnéticas.


\chapter{Conclusão}

Na parte 1, foi possível simular os tempos de transmissão e propagação 
que vimos em sala de aula. Em particular, verificamos como os gráficos 
temporais que mostram os tempos 
de envio e recebimento dos pacotes em forma de barras verticais veriam em 
função do atraso de propagação, largura de banda e tamanho do pacote. Já sabíamos 
que essa dependência existia devido à equação teórica, mas com o NS-3 
foi possível simular isso numericamente com a ferramenta.

Também foi possível na parte 1 usar uma das equações que vimos em sala de aula e comparar 
o valor teórico com o valor simulado, sendo ambos bastante semelhantes.

Na parte 2, aprendemos sobre o papel que nós CSMA desempenham em redes 
cabeadas por Ethernet, evitando colisões ao enviar pacotes ao menos tempo.
Além disso, em sala de aula havíamos visto apenas do ponto de vista teórico as ações 
iniciais que a rede realiza ao estabelecer uma conexão (criar sockets, conectar-se à porta).
Com o NS-3, vimos como isso afeta o atraso fim-a-fim do primeiro pacote, 
estabilizando-se nos pacotes seguintes. 

Por fim, na parte 3 vimos como a mudança no meio de propagação de Ethernet 
para Wi-Fi afeta os atrasos nos envios de pacotes. Vimos também como todos 
os nós de uma rede usam o ponto de acesso (AP) da rede para comunicar-se com outra,
algo que havíamos visto apenas do ponto de vista teórico em sala de aula, 
sendo possível verificar isso na prática com o NS-3.







